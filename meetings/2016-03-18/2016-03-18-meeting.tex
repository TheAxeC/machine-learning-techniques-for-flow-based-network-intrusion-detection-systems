\documentclass[notitlepage]{article}

\usepackage{titling}

\usepackage{graphicx}
\usepackage{float}
\usepackage{rotating}
\usepackage[backend=bibtex, sorting=none]{biblatex}
\usepackage{listings}
\usepackage{parskip}

\posttitle{\par\end{center}}
\setlength{\droptitle}{-80pt}

\title{Meeting}
\author{Axel Faes - 1334986}
\date{Mar 18, 2016}

\begin{document}
\maketitle

aanwezigen: Bram Bonne, Pieter Robyns, Axel Faes

Deze week is voornamelijk besteed aan implementatie. Er zijn verschillende algoritmes ge\"implementeerd. Deze algoritmes zijn K-Nearest Neighbors, K-Means, Lineair Kernel Support Vector Machines met One vs All classification, Lineair Kernel Support Vector Machines met One vs One classification en Gaussian Kernel Support Vector Machines. \\
\\
Er zijn verschillende experimenten uitgevoerd. De poort nummers zijn ge\"implementeerd in 2 varianten. De eerste variant splitst de poorten in categori\"en (veel gebruikte poort nummers en niet-veel gebruikte poort nummers) als een binaire feature. De andere variant maakt van elk poort nummer een nieuwe binaire feature (die stelt of de poort gebruikt is of niet). Deze variant geeft echter heel veel features, dit is enorm inefficient. \\
\\
De gestelde feedback was om te kijken hoe goed het werkt als poort nummers als een continue feature voorgesteld wordt en om de categori\"en op te spliten in $>$1024 en $<$1024. De IP-data can opgesplitst worden in aparte features voor IPv6, IPv4 en MAC. Deze data kan mogelijks voorgesteld worden als continue data. Er is voorgesteld om ook andere algoritmes te implementeren. Om gebruik te kunnen maken van datasets van Cegeka moet ik een presentatie maken en een afspraak maken met professor Quax. \\
\\
Er is feedback gegeven op de thesistekst. Ik moet als eerste goed letten op spelfouten. De inleiding moet algemener uitgelegd worden. Ook niet gebruikte concepter zoals Signature-based IDS moet dieper uitgelegd worden. De attack classification maakt al teveel assumpties over wat er gedetecteerd kan worden. Dit zou eerst algemener uitgelegd moeten worden. Het machine learning hoofstuk bevat in het algemeen te weinig high-level beschrijvingen. 

De actiepunten die gedaan zijn:
\begin{itemize}  
		\item Begin van hoe flowdata gebruikt kan worden
        \item Implementatie van K-Nearest Neighbors
        \item Implementatie van K-Means
        \item Implementatie van Lineair Kernel Support Vector Machines met One vs All classification
        \item Implementatie van Lineair Kernel Support Vector Machines met One vs One classification
        \item Implementatie van Gaussian Kernel Support Vector Machines.
\end{itemize}

Volgende actiepunten zijn besproken:
\begin{itemize}  		
		\item Herschrijven en verwerken van feedback op de thesistekst 
        \item Verdere implementatie: andere algoritmes
        \item Verdere implementatie: Ports indelen in $>$1024 en $<$1024
        \item Verdere implementatie: Ports indelen als continue data
        \item Verdere implementatie: IP indelen als continue data
        \item Verdere implementatie: Starttime instellen als unix time
        \item Maken presentatie voor Cegeka data set
\end{itemize}

\end{document}