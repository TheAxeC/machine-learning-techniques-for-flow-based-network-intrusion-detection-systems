\documentclass[notitlepage]{article}

\usepackage{titling}

\usepackage{graphicx}
\usepackage{float}
\usepackage{rotating}
\usepackage[backend=bibtex, sorting=none]{biblatex}
\usepackage{listings}
\usepackage{parskip}

\posttitle{\par\end{center}}
\setlength{\droptitle}{-80pt}

\title{Meeting}
\author{Axel Faes - 1334986}
\date{Feb 09, 2016}

\begin{document}
\maketitle

aanwezigen: Peter Quax, Bram Bonne, Pieter Robyns, Axel Faes

Dit is de eerste bijeenkomst met de begeleiders en promotor. Er is dus geen rapportering mogelijk van een vorige bijeenkomst. Tijdens de bijeenkomst is beslist om een \textit{intruder detection system} te bestuderen en te implementeren. 

De actiepunten die gedaan moeten worden:
\begin{itemize}  
        \item Beslissen voor wie het systeem gemaakt moet worden. Gaat dit voor end users zijn, of voor grote data centers. Hieraan hangt vast welke data (packets of netflow) gebruikt moet worden.
        \item Bekijken hoe machine learning algoritmes gebruikt kunnen worden in een \textit{intruder detection system}.
        \item Bekijken wat netflow is.
        \item Er moet gekeken worden naar de manier waarop anomalies gegenereerd gaan worden om het systeem te testen/trainen.
\end{itemize}

Volgende afspraken zijn gemaakt:
\begin{itemize}  
		\item Er is gevraagd om te zorgen dat het systeem ook op correcte wijze informatie kan weergeven aan gebruikers. Tijdens het semester moet bekeken worden hoe deze weergave moet gebeuren.
        \item Libraries gebruiken indien mogelijk, om te vermijden dat het wiel opnieuw uitgevonden word.
        \item Er is de mogelijkheid geboden om aan de thesis te werken op het EDM.
        \item Er is afgesproken om \textit{Overleaf} te gebruiken om de thesis in te schrijven.
        \item Een ruwe planning voor het werk moet gemaakt worden tegen 12 Feb.
        \item Een wekelijkse meeting is vastgelegd. Dit om 10:00 elke vrijdag.
        \item Begin mei moet een eerst draft van de thesis klaar zijn en eind mei moet de finale draft af zijn. 
        \item Er moet een vulgariserende tekst gemaakt worden en een postersessie gegeven worden (op 29 juni).
\end{itemize}


\end{document}