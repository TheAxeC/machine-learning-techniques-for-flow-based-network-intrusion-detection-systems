\documentclass[notitlepage]{article}

\usepackage{titling}

\usepackage{graphicx}
\usepackage{float}
\usepackage{rotating}
\usepackage[backend=bibtex, sorting=none]{biblatex}
\usepackage{listings}
\usepackage{parskip}

\posttitle{\par\end{center}}
\setlength{\droptitle}{-80pt}

\title{Meeting}
\author{Axel Faes - 1334986}
\date{Mei 04, 2016}

\begin{document}
\maketitle

aanwezigen: Bram Bonne, Axel Faes \\
\\
Er is feedback gegeven over de machine learning hoofdstukken. De inleiding is goed geschreven. Bij het gedeelte van Feature scaling moet ik 2 voorbeelden omwisselen. Bij de cost function voor logistic regression zou nog een kort voorbeeldje bij moeten. De uitleg over kernels bij Support Vector Machines moet duidelijker verwoord worden. Het gedeelte over distances kan verplaatst worden naar een ander hoofdstuk. Bij Clustering zou pseudocode toegevoegd mogen worden. Decision tree algorithms en Bayesian algorithms moet beter uitgelegd worden. Verder zijn er nog enkele kleinere opmerkingen. Er wordt geprobeerd om al deze feedback te verwerken voor de deadline van de eerste draft (16 mei 2016)\\
\\
De actiepunten die gedaan zijn:
\begin{itemize}  
		\item Schrijven hoofdstuk IP Flows
		\item Afmaken Attack Classification hoofdstuk
		\item Herwerken van hoofdstuk "Machine learning for an IDS"
        \item Verder verwerken dataset Cegeka
\end{itemize}

Volgende actiepunten zijn besproken:
\begin{itemize}  		
		\item Feedback verwerken van machine learning hoofdstukken
		\item Afmaken "implementatie" hoofdstuk
		\item Afmaken "evaluatie" hoofdstuk
		\item Afmaken "Intrusion detection systems" hoofdstuk
		\item Schrijven inleiding hoofdstuk
		\item Schrijven conclusie hoofdstuk
		\item Schrijven Future work hoofdstuk
		\item Schrijven van Nederlandstalige samenvatting
        \item Verder verwerken dataset Cegeka
\end{itemize}


\end{document}