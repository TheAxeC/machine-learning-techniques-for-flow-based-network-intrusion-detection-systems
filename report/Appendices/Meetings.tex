% Appendix A

\chapter{Meetings} % Main appendix title

\label{AppendixA} % For referencing this appendix elsewhere, use \ref{AppendixA}

\section{Meeting 1: 09 Feb 2016}

aanwezigen: Peter Quax, Bram Bonne, Pieter Robyns, Axel Faes\\
\\
Dit is de eerste bijeenkomst met de begeleiders en promotor. Er is dus geen rapportering mogelijk van een vorige bijeenkomst. Tijdens de bijeenkomst is beslist om een \textit{intruder detection system} te bestuderen en te implementeren. \\
\\
De actiepunten die gedaan moeten worden:
\begin{itemize}  
        \item Beslissen voor wie het systeem gemaakt moet worden. Gaat dit voor end users zijn, of voor grote data centers. Hieraan hangt vast welke data (packets of netflow) gebruikt moet worden.
        \item Bekijken hoe machine learning algoritmes gebruikt kunnen worden in een \textit{intruder detection system}.
        \item Bekijken wat netflow is.
        \item Er moet gekeken worden naar de manier waarop anomalies gegenereerd gaan worden om het systeem te testen/trainen.
\end{itemize}

\noindent Volgende afspraken zijn gemaakt:
\begin{itemize}  
		\item Er is gevraagd om te zorgen dat het systeem ook op correcte wijze informatie kan weergeven aan gebruikers. Tijdens het semester moet bekeken worden hoe deze weergave moet gebeuren.
        \item Libraries gebruiken indien mogelijk, om te vermijden dat het wiel opnieuw uitgevonden word.
        \item Er is de mogelijkheid geboden om aan de thesis te werken op het EDM.
        \item Er is afgesproken om \textit{Overleaf} te gebruiken om de thesis in te schrijven.
        \item Een ruwe planning voor het werk moet gemaakt worden tegen 12 Feb.
        \item Een wekelijkse meeting is vastgelegd. Dit om 10:00 elke vrijdag.
        \item Begin mei moet een eerst draft van de thesis klaar zijn en eind mei moet de finale draft af zijn. 
        \item Er moet een vulgariserende tekst gemaakt worden en een postersessie gegeven worden (op 29 juni).
\end{itemize}
\section{Meeting 2: 12 Feb 2016}
aanwezigen: Bram Bonne, Pieter Robyns, Axel Faes\\

\noindent Dit is de tweede bijeenkomst met mijn begeleider. Netflow bevat op zichzelf niet zoveel informatie, maar het is toch handig om te kijken welke bevindingen gemaakt kunnen worden met deze data. Mogelijks kan er, indien gevonden wordt dat netflow alleen niet genoeg informatie bevat, ook gebruikt gemaakt worden van packet data. \\

\noindent Er is de mogelijkheid besproken om eventueel meerdere machine learning algoritmes te implementeren en te bekijken in welke situaties welke algoritmes beter werken.\\

\noindent De actiepunten die gedaan zijn:
\begin{itemize}  
        \item \textit{Beslissen voor wie het systeem gemaakt moet worden.}: Dit gaat gedaan worden voor data centers
        \item Er zijn verschillende classificaties van machine learning algorithmes gevonden die gebruikt kunnen worden.
        \item Verschillende grote data sets van netflow en packets met sporen van anomalies zijn gevonden. Alsook programma's om verkeer te genereren.
\end{itemize}

\noindent Volgende actiepunten zijn besproken:
\begin{itemize}  
		\item Verder uitwerken van welke machine learning algoritmes gebruikt kunnen worden
        \item Bekijken netflow v9
\end{itemize}
\section{Meeting 3: 19 Feb 2016}
aanwezigen: Bram Bonne, Pieter Robyns, Axel Faes\\

\noindent Professor Quax is aan het bekijken ofdat ik (gelabelde) netflow data kan verkrijgen van Cegeka. Dit zou heel handig zijn om mijn implementatie te testen op real world data.\\

\noindent Voorlopig moet ik enkel focussen op een passive intrusion detection systeem, geen preventie en niet direct inline in het netwerkverkeer. Ook de visualisatie moet later bekeken worden, de gebruiker is een netwerkadministrator. Er is tevens besproken dat python zelf mogelijks te traag is om packet sniffing op een goede snelheid uit te voeren. Hiervoor zou ik wireshark kunnen gebruiken (of de command line versie). Er is besproken om eventueel zelf datasets te genereren door malware te runnen op een VM of aparte machine.\\

\noindent De datastructuur voor de machine learning algoritmes is bekeken. Ik moet eens bekijken hoe de timestamps van de flowdata gebruikt kunnen worden. Om de effectiviteit (van de machine learning algoritmes) mogelijks te verhogen ga ik eens bekijken of ip-adressen ingedeeld kunnen worden in country-of-origin of iets dergelijks. Dit zou de machine learning algoritmes de mogelijkheid bieden om ook op deze parameter te bekijken of data malicious is of niet.\\

\noindent De actiepunten die gedaan zijn:
\begin{itemize}  
		\item Er is al een basis implementatie uitgewerkt voor het IDS
        \item De netflow structuur is bekeken en er is een datastructuur opgestelt die gefeed kan worden aan verschillende machine learning algoritmes.
        \item Progressie in de machine learning cursus: chapter 3 van de 18.
\end{itemize}

\noindent Volgende actiepunten zijn besproken:
\begin{itemize}  
        \item Beginnen aan de thesis: het schrijven van een hoofdstuk over machine learning en over hoe deze algoritmes toegepast kunnen worden op een intrusion detection systeem.
        \item Verder werken in de machine learning cursus.
        \item Ik moet eens bekijken ofdat ik een programma vind om pcap files om te zetten naar netflow. Anders moet ik dit zelf schrijven.
\end{itemize}

\noindent Ik heb ook een korte planning gemaakt van hoe de thesis eruit zou zien:
\begin{itemize}  
\item Inleiding:
\begin{itemize}  
    \item wat is een IDS
    \item Waarom is er gekozen voor dit type IDS (host vs netwerk)
    \item Waarom voor data centers
    \item Waarom netflow
    \item Waarom machine learning
\end{itemize}
\item Wat is machine learning
\item Hoe passen we machine learning toe op IDE en wat zijn de voor/nadelen
\item Welke machine learning algortimes zijn wel/niet gebruikt
\item Wat zijn de voor/nadelen van netflow
\item Hoe met combinatie netflow/packets (Als dit gedaan zou worden)
\item Welke data sets zijn gebruikt
\item Wat zijn de bevindingen
\item Hoe kan visualisatie/feedback gebeuren (richting admin en richting automatische preventie)
\item Conclusie
\end{itemize}
\section{Meeting 4: 26 Feb 2016}
aanwezigen: Bram Bonne, Axel Faes\\

\noindent Deze week is voornamelijk besteed aan de implementatie. Er is een netflow exporter geschreven. Er is bekeken ofdat timestamps gebruikt kunnen worden en ofdat ip-adressen opgedeeld kunnen worden per land. Er is besloten dat dit zeer weinig effect heeft op de accuraatheid van de machine learning algoritmes.\\

\noindent Momenteel zijn Support vector machines en K-nearest Neighbor Classifier algoritmes bekeken. Het K-nearest Neighbor Classifier algoritme is zeer efficient (~98\%).\\

\noindent In een later stadium kan bekeken worden om eventueel verdere analyse te doen op de data die malicious gevonden is, eventueel door pakketten te analyseren, of nogmaals door machine learning technieken. Er kan ook eens bekeken worden om een VM op te zetten, en daarin malware te runnen en dit verkeer te monitoren. Herbij zouden eigen datasets gegenereerd kunnen worden.\\

\noindent De machine learning cursus is gevolgd tot hoofstuk 7. De cursus zou normaal af moeten zijn binnen 2 weken. \\

\noindent De actiepunten die gedaan zijn:
\begin{itemize}  
		\item Er is al een netflow exporter geschreven
        \item Er zijn experimenten uitgevoerd m.b.t de datastructuur die meegegeven wordt aan de machine learning cursus.
        \item Progressie in de machine learning cursus: chapter 7 van de 18.
        \item Er is begonnen aan de thesis. 
        \item Het zou interessant zijn om eens te kijken ofdat ip-addressen opgedeeld kunnen worden in subnets.
\end{itemize}

\noindent Volgende actiepunten zijn besproken:
\begin{itemize}  
        \item Focussen op de thesis
        \item Verder werken in de machine learning cursus.
\end{itemize}

\section{Meeting 5: 04 Mar 2016}
aanwezigen: Bram Bonne, Axel Faes\\

\noindent Deze week is voornamelijk besteed aan de thesis en aan het leren van de machine learning cursus. De machine learning cursus is gevolgd tot hoofstuk 10. De cursus zou tegen volgende meeting af moeten zijn. De algemene structuur van de thesistekst is nagekeken. Het hoofdstuk over "Attack classification" moet uitgebreid worden met een algemene uitleg over hoe aanvallen gedetecteerd kunnen worden. Het hoofdstuk over de gebruikte data-sets moet samengevoegd worden met het hoofdstuk dat de implementatie beschrijft. Het hoofdstuk over voor/nadelen van machine learning voor intrusion detection systemen moet verwerkt worden in het algemene hoofdstuk over machine learning.\\

\noindent De actiepunten die gedaan zijn:
\begin{itemize}  
		\item Verder werken aan ML cursus
        \item Schrijven aan thesis.
\end{itemize}

\noindent Volgende actiepunten zijn besproken:
\begin{itemize}  		
\item Verder werken aan ML cursus
        \item Schrijven aan thesis.
\end{itemize}

\section{Tussentijdse presentatie: 08 Mar 2016}
aanwezigen: Maarten Wijnants, Peter Quax, Wim Lamotte, Jori Liesenborgs, Wouter vanmontfort, Pieter Robyns, Robin Marx, Bram Bonne, Axel Faes\\

\noindent Er is een tussentijdse presentatie geweest waarbij ik mijn huidige progressie moest tonen en een planning moest geven. De presentatie zelf is goed verlopen. Na de presentatie heb ik verschillende vragen gekregen. \\

\noindent Veel vragen die gesteld waren, waren bedoeld om te kijken of we het nut/doel van de bachelorthesis kennen en hoe we de invulling correct doen. Ook een belangrijk aspect is hoe het valideren van de correctheid van de experimenten die gedaan zijn/worden zal gebeuren. \\

\noindent Een opmerking was dat ik ook bestaande Intrusion detection systemen moet bekijken en ofdat deze machine learning gebruiken. Dan is ook belangrijk waarom ze het wel of niet gebruiken. \\

\noindent Er was verwacht dat ik al iets verder stond met de Machine learning cursus. Hierdoor kon ik niet altijd op de volledige diepgang de gestelde vragen beantwoorden. Ik begreep ook niet altijd de onderliggende vraag waardoor ik te oppervlakkig antwoorde. Qua machine learning algoritmes moest ik goed opletten voor overfitting en uitleggen hoe ik hiermee omga.\\

\noindent Er zijn ook vragen gesteld m.b.t mijn geplande extra om het intrusion detection systeem real-time te maken. Normaal wordt een flow pas doorgegeven als deze volledig afgesloten is, een mogelijke piste zou zijn om flows al te bekijken ook al zijn ze nog niet afgesloten. Ook het runnen van een VM met malware erop om zelf data-sets te generen is bevestigd dat een goed idee zou zijn. Als ik hiervoor infrastructuur nodig heb moet ik dit vragen.\\

\noindent Ik moet opletten met aanvallen die maar zeer weinig netwerktraffiek genereren. Ook moet ik goed beschrijven welke aanvallen wel of niet gedetecteert kunnen worden en uitleggen waarom. Dit staat momenteel al beschreven in mijn thesistekst. Ik zou ook mogelijkheden kunnen uitleggen die ervoor zouden kunnen zorgen dat ik toch alle (of een groot deel) van de aanvallen zou kunnen detecteren. Dit zou bv kunnen door toch packet-data te gaan bekijken.\\

\noindent Een algemene opmerking die gegeven was, was dat de presentatie visueler mocht zijn. Figuren en afbeeldingen zijn aangenamer om te tonen aan een publiek. Bij de postersessie moet er ook goed opgelet worden dat ik van persoon tot persoon bekijk hoe diep ik de materie uit mijn bachelorthesis kan uitleggen. 

\section{Meeting 6: 11 Mar 2016}
aanwezigen: Bram Bonne, Axel Faes\\\\
\noindent Deze week is voornamelijk besteed aan de thesis en het afmaken van de machine learning cursus. De machine learning cursus is volgens planning afgemaakt. De thesistekst moet ingestuurd worden zodat deze al nagekeken kan worden. Komende weken zal gespendeerd worden aan de implementatie en het uitzoeken hoe de flowdata gebruikt kan worden in de machine learning algoritmes.

\noindent De actiepunten die gedaan zijn:
\begin{itemize}  
		\item Afmaken Machine learning cursus
        \item Schrijven aan thesis.
\end{itemize}

\noindent Volgende actiepunten zijn besproken:
\begin{itemize}  		
		\item Verdere implementatie algoritmes
        \item Bekijken hoe de flowdata gefeed kan worden aan de machine learning algoritmes
\end{itemize}

\section{Meeting 7: 18 Mar 2016}
aanwezigen: Bram Bonne, Pieter Robyns, Axel Faes\\\\
Deze week is voornamelijk besteed aan implementatie. Er zijn verschillende algoritmes ge\"implementeerd. Deze algoritmes zijn K-Nearest Neighbors, K-Means, Lineair Kernel Support Vector Machines met One vs All classification, Lineair Kernel Support Vector Machines met One vs One classification en Gaussian Kernel Support Vector Machines. \\
\\
Er zijn verschillende experimenten uitgevoerd. De poort nummers zijn ge\"implementeerd in 2 varianten. De eerste variant splitst de poorten in categori\"en (veel gebruikte poort nummers en niet-veel gebruikte poort nummers) als een binaire feature. De andere variant maakt van elk poort nummer een nieuwe binaire feature (die stelt of de poort gebruikt is of niet). Deze variant geeft echter heel veel features, dit is enorm inefficient. \\
\\
De gestelde feedback was om te kijken hoe goed het werkt als poort nummers als een continue feature voorgesteld wordt en om de categori\"en op te spliten in $>$1024 en $<$1024. De IP-data can opgesplitst worden in aparte features voor IPv6, IPv4 en MAC. Deze data kan mogelijks voorgesteld worden als continue data. Er is voorgesteld om ook andere algoritmes te implementeren. Om gebruik te kunnen maken van datasets van Cegeka moet ik een presentatie maken en een afspraak maken met professor Quax. \\
\\
Er is feedback gegeven op de thesistekst. Ik moet als eerste goed letten op spelfouten. De inleiding moet algemener uitgelegd worden. Ook niet gebruikte concepter zoals Signature-based IDS moet dieper uitgelegd worden. De attack classification maakt al teveel assumpties over wat er gedetecteerd kan worden. Dit zou eerst algemener uitgelegd moeten worden. Het machine learning hoofstuk bevat in het algemeen te weinig high-level beschrijvingen. 

De actiepunten die gedaan zijn:
\begin{itemize}  
		\item Begin van hoe flowdata gebruikt kan worden
        \item Implementatie van K-Nearest Neighbors
        \item Implementatie van K-Means
        \item Implementatie van Lineair Kernel Support Vector Machines met One vs All classification
        \item Implementatie van Lineair Kernel Support Vector Machines met One vs One classification
        \item Implementatie van Gaussian Kernel Support Vector Machines.
\end{itemize}

Volgende actiepunten zijn besproken:
\begin{itemize}  		
		\item Herschrijven en verwerken van feedback op de thesistekst 
        \item Verdere implementatie: andere algoritmes
        \item Verdere implementatie: Ports indelen in $>$1024 en $<$1024
        \item Verdere implementatie: Ports indelen als continue data
        \item Verdere implementatie: IP indelen als continue data
        \item Verdere implementatie: Starttime instellen als unix time
        \item Maken presentatie voor Cegeka data set
\end{itemize}
\section{Meeting 8: 24 Mar 2016}
aanwezigen: Bram Bonne, Pieter Robyns, Axel Faes\\\\
Deze week is voornamelijk besteed aan implementatie. Er zijn verschillende experimenten uitgevoerd. De poort nummers zijn ge\"implementeerd zodanig dat er een indeling is tussen normale poorten ($<$1024) en speciale poorten ($>$1024). Er is ook gekeken ofdat poort data niet continu voorgesteld kan worden. Dit geeft echter slechtere resultaten dan te werken met de binaire indeling.\\\\
De IP-data kan opgesplitst worden in aparte features voor IPv6, IPv4 en MAC. Deze data is voorgesteld als continue data. Het is moeilijk om IP-addressen zelf te gaan onderverdelen in categorieen of subnets. Het indelen als continue data geeft dan ook de beste accuraatheid.  De starttijd is ook ge\"implementeerd om te voegen als feature. Dit gebeurt door te bekijken in welke dag van de week/ uur van de dag en minuut van het uur de data gegenereerd word. Echter had deze feature geen goede invloed op de accuraatheid\\
\\
Er zijn enkele algoritmes verder ge\"implementeerd zoals een Decision Tree learner en een Naive Bayes algoritmes. Beide zijn supervised learning technieken. De Naive Bayes had nog goede accuraatheid, de Decision Tree Learner gaf slechtere resultaten.

De actiepunten die gedaan zijn:
\begin{itemize}  
        \item Verdere implementatie: andere algoritmes
        \item Verdere implementatie: Ports indelen in $>$1024 en $<$1024
        \item Verdere implementatie: Ports indelen als continue data
        \item Verdere implementatie: IP indelen als continue data
        \item Verdere implementatie: Starttime instellen als unix time
\end{itemize}

Volgende actiepunten zijn besproken:
\begin{itemize}  		
		\item Herschrijven en verwerken van feedback op de thesistekst 
        \item Testen van de implementatie
        \item Bekijken van Unsupervised learning algoritmes
\end{itemize}
\section{Meeting 9: 01 April 2016}
aanwezigen: Bram Bonne, Peter Quax, Axel Faes\\\\
Deze week is voornamelijk besteed aan implementatie. Er is een nieuwe dataset gevonden. Deze dataset is afkomstig van de universiteit van Twente. Er was een honeypot opgesteld op het netwerk, alle data die hiermee gevangen is, is geclassificeerd en een dataset mee gemaakt. De dataset bevat voornamelijk externe aanvallen. \\\\
Er is ook gefocused op het trachten te gebruiken van unsupervised learning algoritmes. Echter heeft dit weinig opgebracht. Er kan wel op een accurate manier onderscheidt gemaakt worden tussen malicious en niet malicious data, maar het is moeilijk om vervolgens af te leiden over welk type malicious data het gaat. \\\\
De features die gebruikt worden in de machine learning algoritmes zijn nog eens overlopen. Professor Quax kwam met het idee om IP-adressen eventueel op te delen in origine (zoals land). Dit kan gebeuren via services zoals WhoIs. \\\\
De EDM dataet kan afgehaald worden bij het kantoor van professor Quax. Er moet ook zo snel mogelijk een meeting georganiseerd worden met Cegeka.

De actiepunten die gedaan zijn:
\begin{itemize}  
		\item Herschrijven en verwerken van feedback op de thesistekst 
        \item Testen van de implementatie
        \item Bekijken van Unsupervised learning algoritmes
\end{itemize}

Volgende actiepunten zijn besproken:
\begin{itemize}  		
		\item Verwerken data EDM
		\item Implementatie van WhoIs als feature
       \item Maken presentatie voor Cegeka data set
\end{itemize}
\section{Meeting Cegeka: 06 April 2016}
aanwezigen: Peter Quax, Axel Faes, Cegeka\\\\
Er is een meeting geweest met Cegeka om de mogelijkheid te bespreken om data te kunnen gebruiken die afkomstig is van Cegeka, alsook om informatie te krijgen van de hudige IDS' die gebruikt worden. \\\\
In het begin is een korte presentatie gegeven waarin de eigenschappen van het te ontwikkelen systeem uitgelegd worden. Er wordt ook beschreven wat de algemene onderzoeksvragen zijn en hoe data nu gebruikt kan worden in machine learning algoritmes. \\\\
Een eerste vraag die gesteld is, is welke classificatie van ‘onverwachte traffiek’ er momenteel gebeurt in datacenter/hosting context. Cegeka werkt heel gelaagd. Voor de routers staat een DDoS protection systeem. Na de router staan zowel firewalls als SIEM devices. Voor grotere klanten is er extra beveiliging voorzien in de vorm van afgestelde IPS systemen en meer gedetailleerde threat detection. Zoveel mogelijk data wordt verwerkt, zowel flows als meer gedetailleerd. De systemen werken voornamelijk met een signature database en verwerken de data voornamelijk automatisch. De systemen moeten wel manueel afgesteld en onderhouden worden. \\\\
De classificatie gebeurt heel gedetailleerd door deze verschillende lagen. Er werd wel gesteld dat het veel interessanter is om outbound verkeer na te kijken in vergelijking met binnengaand verkeer. Zaken zoals port-scans zijn interessant om te weten maar gebeuren heel veel en kunnen al goed gedetecteerd worden. \\\\
Er was veel interesse naar een intrusion detectie systeem dat op basis van netflow en machine learning technieken werkt. Er accuraatheid van rond de 70-80 procent zou gezien worden als een goede accuraatheid. Ze hebben ook liever false positives dan false negatives. Teveel alerts genereren is heel vervelend, maar het is belangrijker dat voldoende anomalieen gedetecteerd worden. \\\\
Er is gevraagd of het mogelijk is om data te verkrijgen. Dit was zeker mogelijk. De netflow en corresponderende logs van 3 dagen wordt geleverd. De logs worden zowel in binair als text formaat geleverd. Het binair formaat kan ingelezen door een programma dat de logs visualiseerd. Dit programma wordt ook meegeleverd. Mogelijks zou ook output van het DDoS systeem geleverd kunnen worden. Om te bekijken ofdat een flow gezien is als malicious of niet moet dit bekeken worden of deze flow voorkomt in de logs.
\section{Meeting 10: 12 April 2016}
aanwezigen: Bram Bonne, Axel Faes\\\\
Afgelopen week, woensdag 6 april, is de meeting geweest met Cegeka. Van deze meeting is een verslag gemaakt. Op maandag 4 april is de dataset van het EDM verkregen. Deze dataset bevat netflow data van het EDM netwerk van 18 februari tot 24 maart. Elke dag is netflow beschikbaar die voorgekomen is tussen 10u tot 24u. Deze zijn per 15 minuten gelogd in een file. \\\\
Er is kort gewerkt aan het verwerken van de data van het EDM. Dit gebeurt door de data te laten verwerken door verschillende algoritmes. De data waarvan vervolgens gesteld kan worden dat deze daadwerkelijk malicious is, zal doorgegeven worden aan professor Quax. Op het moment zijn er nog niet genoeg testen uitgevoerd om iets te kunnen zeggen over de data. \\\\
Er is ge\"implementeerd dat de country-of-origin van een IP ook gebruikt kan worden als feature voor de machine learning algoritmes. Er is gevonden dat dit voornamelijk werkt voor data die van buitenaf komt, zoals port scans. Er is ook kort al geprobeerd om visualisaties te maken van de machine learning algoritmes. Zulke visualisaties zouden interessant kunnen zijn om ook in de thesistekst te gebruiken. Tegen volgende bijeenkomst moet voornamelijk gewerkt worden aan de thesistekst.
De actiepunten die gedaan zijn:
\begin{itemize}  
		\item Meeting Cegeka
        \item Implementatie van WhoIs als feature
        \item Implementatie van kleine visualisatie van algoritmes
        \item  Verkrijgen data van EDM
\end{itemize}

Volgende actiepunten zijn besproken:
\begin{itemize}  		
		\item Herschrijven en verwerken van feedback op de thesistekst
        \item Visualisaties van machine learning algoritmes zijn interessant voor thesistekst.
\end{itemize}
\section{Meeting 11: 18 April 2016}
aanwezigen: Bram Bonne, Axel Faes \\
\\
Ik heb mijn huidige vooruitgang laten zien van de bachelorproef tekst. Hierbij is een korte herschikking gekomen van de hoofdstukken. Ik moet het "Implementatie" hoofdstuk opsplitsen naar een hoofstuk "Implementatie" en een hoofdstuk "Evaluatie". In "Implementatie" moet mijn implementatie zelf beschreven staan, in "Evaluatie" moeten de resultaten beschreven worden. \\\\
Verder is kort uitgelegd wat de algemene structuur is van het machine learning hoofdstuk en welke aanpassingen er gebeurd zijn. Het hoofdstuk is opgesplitst in meerdere delen zodanig dat de uitleg, de algoritmes en validatie in aparte hoofdstukken uitgelegd wordt. Het hoofdstuk "Preventie" moet ik aanpassen naar iets gelijkaardigs aan "Future work". Het hoofdstuk "Visualisatie" moet samengevoegd worden met "Implementatie".
De actiepunten die gedaan zijn:
\begin{itemize}  
        \item  Verwerken feedback op machine learning hoofdstukken.
\end{itemize}

Volgende actiepunten zijn besproken:
\begin{itemize}  		
		\item Herschrijven en verwerken van feedback op de thesistekst
        \item Halen data bij Cegeka
\end{itemize}
\section{Meeting 12: 22 April 2016}
aanwezigen: Bram Bonne, Axel Faes \\
\\
Ik heb mijn huidige vooruitgang laten zien van de bachelorproef tekst. De meeste tekst is al stukken beter. Er mogen bij enkele stukken nog numerieke voorbeelden komen te staan. Deze stukken zijn de regularisatie en de feature scaling. Het hoofdstuk over "Algorithms" moet nog herschreven worden. Tevens ga ik een "Summary" schrijven op het einde van het machine learning hoofdstuk zodanig dat nog snel en kort een samenvatting gegeven wordt.\\
\\
De actiepunten die gedaan zijn:
\begin{itemize}  
        \item  Verwerken feedback op machine learning hoofdstukken.
\end{itemize}

Volgende actiepunten zijn besproken:
\begin{itemize}  		
		\item Verwerken feedback en herschrijven "Algorithms" tegen zondag 24 April
        \item Schrijven van tekst
\end{itemize}

\section{Meeting 13: 29 April 2016}
aanwezigen: Bram Bonne, Axel Faes \\
\\
Het herschrijven van het "Algorithms" hoofdstuk heeft langer geduurd dan gedacht en is pas ingeleverd op woensdag 27 april 2016. Tijdens de meeting is toegelicht welke aanpassingen gebeurt zijn aan de thesis. Er is kort besproken dat er ook focus gelegd moet worden over het verwerken van de data van Cegeka. Dit zou al belangrijk zijn voor de eerste draft van de thesis. \\
\\
De actiepunten die gedaan zijn:
\begin{itemize}  
        \item Beginnen met verwerken van data van Cegeka
        \item Schrijven korte summary op het einde van het machine learning hoofdstuk
        \item Numerieke voorbeelden plaatsen bij regularisatie en feature scaling
        \item Hoofdstuk "Algorithms" herwerken
\end{itemize}

Volgende actiepunten zijn besproken:
\begin{itemize}  		
		\item Schrijven hoofdstuk IP Flows
		\item Herwerken van hoofdstuk "Machine learning for an IDS"
        \item Verder verwerken dataset Cegeka
\end{itemize}

\section{Meeting 14: 04 Mei 2016}
aanwezigen: Bram Bonne, Axel Faes \\
\\
Er is feedback gegeven over de machine learning hoofdstukken. De inleiding is goed geschreven. Bij het gedeelte van Feature scaling moet ik 2 voorbeelden omwisselen. Bij de cost function voor logistic regression zou nog een kort voorbeeldje bij moeten. De uitleg over kernels bij Support Vector Machines moet duidelijker verwoord worden. Het gedeelte over distances kan verplaatst worden naar een ander hoofdstuk. Bij Clustering zou pseudocode toegevoegd mogen worden. Decision tree algorithms en Bayesian algorithms moet beter uitgelegd worden. Verder zijn er nog enkele kleinere opmerkingen. Er wordt geprobeerd om al deze feedback te verwerken voor de deadline van de eerste draft (16 mei 2016)\\
\\
De actiepunten die gedaan zijn:
\begin{itemize}  
		\item Schrijven hoofdstuk IP Flows
		\item Afmaken Attack Classification hoofdstuk
		\item Herwerken van hoofdstuk "Machine learning for an IDS"
        \item Verder verwerken dataset Cegeka
\end{itemize}

Volgende actiepunten zijn besproken:
\begin{itemize}  		
		\item Feedback verwerken van machine learning hoofdstukken
		\item Afmaken "implementatie" hoofdstuk
		\item Afmaken "evaluatie" hoofdstuk
		\item Afmaken "Intrusion detection systems" hoofdstuk
		\item Schrijven inleiding hoofdstuk
		\item Schrijven conclusie hoofdstuk
		\item Schrijven Future work hoofdstuk
		\item Schrijven van Nederlandstalige samenvatting
        \item Verder verwerken dataset Cegeka
\end{itemize}
