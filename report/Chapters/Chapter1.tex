% Chapter 1

\chapter{Introduction} % Main chapter title

\label{Chapter1} % Change X to a consecutive number; for referencing this chapter elsewhere, use \ref{ChapterX}

The internet is constantly growing and new network sevices arise constantly. This has as effect that security flaws become more and more important. Considering this, it becomes more important to be able to detect and prevent attacks on network systems.

\section{Intrusion detection systems}
An intrusion detection system is a system which tries to determine whether a system is under attack, to detect intrusions within a system. There are different types of intrusion detection systems or IDS. There are network-based intrusion detection systems and host-based intrusion detection systems.

\subsection{Network-based Intrusion Detection Systems}
Network-based intrusion detection systems are placed at certain points within a network in order to monitor traffic from and to devices within the network. The system can analyse the traffic using multiple techniques to determine whether the data is malicious. There are two different ways to analyse the network data. The analysis can be packet-based or flow-based.\\
\\
Packet-based analysis uses the entire packet including the headers and payload. The advantage of this type of analysis is that there is a lot of data to work with. Every single byte of the packet could be used to determine whether the packet is malicious or not. The disadvantage is immediately obvious once we look at networks through which a lot of data passes, such as data centers. Analysing every byte is very work-intensive and near impossible to do in such environments.  \\
\\
Flow-based analysis doesn't use individual packets but uses general data about network flows. A flow is defined as a single connection between the host and another device. A flow can be defined using a (source\_IP, destination\_IP, source\_port, destination\_port) tuple. However flowdata also contains other information such as the duration of the connection, the start time, the amount of bytes and/or packets within the flow. Flow data can even contain data such as the amount of SYN packets within the flow. This could be useful to detect SYN overflow attacks. However not every flow collector collects this data.

\subsection{Host-based Intrusion Detection Systems}
Host-based intrusion detection systems are systems that monitor the device on which they are installed. The way they monitor the system can range from monitoring the state of the main system through log files, to monitoring program execution. In this way they can be quite indistinguishable from Anti-Virus programs.

\subsection{Intrusion Prevention Systems}
An intrusion prevention system or IPS/IDPS is an intrusion detection system that also has to ability to prevent attacks. An IDS does not necessarily need to be able to detect attacks at the exact moment they occur, although it is preferred. An IPS needs to be able to detect attacks real-time since it also needs to be able to prevent these attacks. For network attacks these prevention actions could be closing the connection, blocking an IP, limiting the data throughput.