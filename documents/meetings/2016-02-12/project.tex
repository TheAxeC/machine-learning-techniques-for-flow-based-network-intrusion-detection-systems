\documentclass[notitlepage]{article}

\usepackage{titling}

\usepackage{graphicx}
\usepackage{float}
\usepackage{rotating}
\usepackage[backend=bibtex, sorting=none]{biblatex}
\usepackage{listings}
\usepackage{parskip}

\posttitle{\par\end{center}}
\setlength{\droptitle}{-80pt}

\title{Meeting}
\author{Axel Faes - 1334986}
\date{Feb 12, 2016}

\begin{document}
\maketitle

aanwezigen: Bram Bonne, Pieter Robyns, Axel Faes

Dit is de tweede bijeenkomst met mijn begeleider. Netflow bevat op zichzelf niet zoveel informatie, maar het is toch handig om te kijken welke bevindingen gemaakt kunnen worden met deze data. Mogelijks kan er, indien gevonden wordt dat netflow alleen niet genoeg informatie bevat, ook gebruikt gemaakt worden van packet data. 

Er is de mogelijkheid besproken om eventueel meerdere machine learning algoritmes te implementeren en te bekijken in welke situaties welke algoritmes beter werken.

De actiepunten die gedaan zijn:
\begin{itemize}  
        \item \textit{Beslissen voor wie het systeem gemaakt moet worden.}: Dit gaat gedaan worden voor data centers
        \item Er zijn verschillende classificaties van machine learning algorithmes gevonden die gebruikt kunnen worden.
        \item Verschillende grote data sets van netflow en packets met sporen van anomalies zijn gevonden. Alsook programma's om verkeer te genereren.
\end{itemize}

Volgende actiepunten zijn besproken:
\begin{itemize}  
		\item Verder uitwerken van welke machine learning algoritmes gebruikt kunnen worden
        \item Bekijken netflow v9
\end{itemize}


\end{document}