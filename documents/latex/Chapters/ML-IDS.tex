% Chapter 1

\chapter{Machine learning for an IDS} % Main chapter title

\label{Chapter3} % Change X to a consecutive number; for referencing this chapter elsewhere, use \ref{ChapterX}

An intrusion detection system has to detect whether some data it receives is either malicious or regular web traffic. This can be seen as a classification problem which means a machine learning algorithm for classification could be used. It needs to be determined whether data is either normal network traffic or, whether it corresponds to malicious behaviour. \\
\\
Some parameters have to be chosen that will be feed into the machine learning algorithm. These parameters can be log files, entire packets, packet headers or IP flows as seen in Section~\ref{flow}. In general there are several advantages and disadvantages to using machine learning for intrusion detection.

\section{Properties of using ML for an IDS}
As said before, machine learning for an intrusion detection system is a classification problem. More precisely, it can be said that intrusion detection systems have to detect abnormal behaviour in a network with mostly normal behaviour. There are several problems that can be encountered when using machine learning techniques.

\subsection{Network diversity}
A first problem is the diversity of network traffic. The notion of "normal network traffic" is difficult to actually define. The bandwidth, duration of connections, origin of IP addresses, applications used can vary enormously through time. This makes it quite difficult for machine learning algorithms to distinguish between "normal network traffic" and malicious behaviour.\cite{ClosedWorld} \\
\\
However, this issue can be bypassed by defining what malicious behaviour looks like. If a machine learning algorithm learns what malicious behaviour looks like, it does not need to know anything about normal behaviour. 

\subsection{Detect new attacks}
Another problem is the ability to detect new attacks. A machine learning algorithm compares incoming data with a model that it has created internally. An new type of malicious behaviour might appear to be closer to normal network traffic as compared to the model of known attacks. This is only an issue for completely new attacks since most attacks are similar to known malicious behaviour. The model that is created for malicious behaviour can better detect newer attacks than a signature based IDS. \cite{ClosedWorld}

\subsection{Classification}
There is only one big remaining problem when trying to teach a machine learning algorithm what malicious behaviour looks like. Different types of malicious behaviour can look completely different. When the machine learning algorithm constructs a model, the model might not be able to generalise well or it might give a lot of false positives or false negatives. \\
\\
There are several solutions that can be used in order to make machine learning algorithms more effective for intrusion detection systems. One option is to chance the way the classification problem is defined. Instead of defining the classes, "normal" and "malicious", there might be different classes for different types of malicious behaviour. In the same way, different classes can be defined for different types normal traffic. \\
\\
This means that a different class is used for each different type of attack. This also makes it easy to define different severety levels for different attacks. Since the machine learning algorithm outputs the predicted class, a more detailed report can be generated on the anomaly. \cite{ClosedWorld}

\subsection{Learning}
A big advantage of using machine learning is the learning part of the algorithm. The algorithm learns what malicious behaviour looks like. Whenever it has to predict whether something is malicious and what class it belongs to, it does not need to use perfect matching or heuristics such as other forms of intrusion detection as seen in Section~\ref{detection}. It has learned a model that describes what the different classes of malicious behaviour look like. \cite{subaira2014efficient}

\section{Evaluating ML for an IDS}
With a machine learning algorithm, performance can be measured using the F-score as seen in Section~\ref{errorAnalysis}. However, for intrusion detection systems, this is not enough by itself. The F-score assumes that recall and precision has the same importance. This is not necessarily the case when evaluating intrusion detection systems. \cite{subaira2014efficient}\\
\\
In Table~\ref{tab:perform}, it is shown when there is a true positive and true negative. A false positive occurs when a sample is actually Normal but is classified as an Intrusion. A false negative occurs when a sample is actually an Intrusion but is classified as Normal. \\
\\
\begin{table}[H]
\caption{Performance measure.}
\label{tab:perform}
\centering
\begin{tabular}{l c r}
\toprule
\tabhead{} & \tabhead{Intrusion} & \tabhead{Normal}\\
\midrule
\textbf{Intrusion} & True Positive & False Negative \\
\midrule
\textbf{Normal} & False Positive & True Negative\\
\bottomrule
\end{tabular}
\end{table}

\noindent A false negative is bad, since it means that an Intrusion was not detected. But most IDS's are used in a layered approach. This means that if one layer does not detect an Intrusion, another layer might detect it. \\
\\
The top layer usually looks at the entire network traffic. Because the entire network traffic might be huge, the data that is processed is kept to a minimum. This means that the top layer will use IP flows. This means that not detecting an Intrusion is not that horrible. A low recall could be disastrous if the amount of detected Intrusions is huge, which can happen if the amount of data that is passed through the IDS is also huge. \\
\\
The layered approach might also work completely different. Perhaps the first layer tries to detect as many anomalies as possible (and having a low recall) and then passing the data for which anomalies have been detected to other layers. This approach means that a low recall is not bad.  \\
\\
The scoring used for an IDS that uses machine learning is dependant on how the IDS is going to be used. \cite{subaira2014efficient}

\section{Using ML for an IDS}
Data has to be processed before it can be used within a machine learning algorithm. This means that features have to be chosen. Some features can be easy to find, other have to found by experimenting and running tests. \\
\\
Using all the features of a dataset does not necessarily guarantee the best
performances from the IDS. It might increase the computational cost as well as the
error rate of the system. This is because some features are redundant or are not usefull for making a distinction between different classes. \cite{al2014machine}
